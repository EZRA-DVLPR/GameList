\section{Legal}

As with anything released, there must be some form of legal
binding/documentation to ensure things are handled properly and
legally for both developers and the users of the programs.

I have separated the license into 2 parts:

\begin{enumerate}
	\item The Icon and Name
	\item The rest of the program + documentation
\end{enumerate}

This is FOSS (Free and Open Source Software). I intend to keep the project
this way. I also encourage others to remix, modify, and
look at the repo. However, I think that these same people shouldn't be
able to re-release the software under the same name/icon and say it
was their initial idea. As such, I have chosen a license for
attribution to protect the concept of the name and icon, while the
rest is under the
MIT license to basically allow anything to happen.

Even as a developer (even an adult in general), it's important to
understand the basics of legalese (legal speak and terminology).
Much of the adult world is governed by these rules, and to be
ignorant of how it operates is a choice. This may be harsh, but such is
life. Just because someone didn't read the fine print, nor understood the
consequences beforehand does not grant one clemency from consequences
of their actions.

I opted for open source licenses and ones that were relatively simple
to understand. I will describe each and then explain the meaning in
more common language. The hope is to show a reasoning behind why
these licenses were picked in particular for the portion of the project.

\subsection{Icon and Name}

Below is a copy of the license for the icon and name explicitly. I
will explain it after. The reason for this is to protect the icon and
name from unjust attribution should someone want to re-distribute this program.

\newpage

\begin{verbatim}
# License for Icon and Name

The icon and name associated with this project are licensed
under the Creative Commons Attribution 4.0 International
License (CC BY 4.0).

You are free to:
- **Share** — copy and redistribute the material in any
medium or format
- **Adapt** — remix, transform, and build upon the material
for any purpose, even commercially.

Under the following terms:
- **Attribution** — You must give appropriate credit, provide
a link to the license, and indicate if changes were made. You
may do so in any reasonable manner, but not in any way that
suggests the licensor endorses you or your use.

Full License:
[https://creativecommons.org/licenses/by/4.0/]
(https://creativecommons.org/licenses/by/4.0/)
\end{verbatim}

\newpage

The license explicitly states that anyone is capable
of copying, redistributing, remixing, transforming, modifying, and
building upon the icon and name. This can be released for free or
be sold. My software is free, but if someone who wants to build upon
it, remix and release for a cost, then that is their choice and I
won't restrict that.

There is a single restriction, that is the requirement to give
credit. If someone makes changes to the name or icon, then they must
state them in a clear manner to the user of the modified software.
They must cite their source
(me, EZRA-DVLPR), and then say what they changed. Furthermore, it
must be understood that I (EZRA-DVLPR), do not endorse this new
software by any means.

\subsection{The Rest}

This license covers the rest of the project and repository as a
whole. Basically the code, documentation, manual etc. The hope is to
allow anyone to modify and remix this part however they see fit. I
really don't mind what they do with it to be honest.

Once again, I will print the license verbatim, then follow it up with
an explanation.

\newpage

\begin{verbatim}
MIT License

Copyright (c) 2025 EZRA-DVLPR

Permission is hereby granted, free of charge, to any person
obtaining a copy of this software and associated documentation
files (the "Software"), to deal in the Software without
restriction, including without limitation the rights to use,
copy, modify, merge, publish, distribute, sublicense, and/or
sell copies of the Software, and to permit persons to whom the
Software is furnished to do so, subject to the following
conditions:

The above copyright notice and this permission notice shall
be included in all copies or substantial portions of the Software.

THE SOFTWARE IS PROVIDED "AS IS", WITHOUT WARRANTY OF ANY
KIND, EXPRESS OR IMPLIED, INCLUDING BUT NOT LIMITED TO THE
WARRANTIES OF MERCHANTABILITY, FITNESS FOR A PARTICULAR PURPOSE
AND NONINFRINGEMENT. IN NO EVENT SHALL THE AUTHORS OR COPYRIGHT
HOLDERS BE LIABLE FOR ANY CLAIM, DAMAGES OR OTHER LIABILITY,
WHETHER IN AN ACTION OF CONTRACT, TORT OR OTHERWISE, ARISING FROM,
OUT OF OR IN CONNECTION WITH THE SOFTWARE OR THE USE OR OTHER
DEALINGS IN THE SOFTWARE.
\end{verbatim}

\newpage

The MIT License is very simple and probably the most used one for
software. It says that the software is given for free and allows
anyone to obtain a copy and do anything they please with it. This
includes selling or redistributing it.
It also says that the software is provided as-is meaning if there are
any bugs or problems, then users acknowledge and agree to it regardless.

Another advantage of this license is the ability to make changes, and
still be upheld. For example, by adding a new piece of functionality,
I can just repackage the program and re-release it without issue.
