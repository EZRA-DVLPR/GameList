\section{Logical Design}
\label{sec:LogicalDesign}

The idea behind the logical design is to describe what the program
looks like in terms of logical units/functions/code portions. This
means it gets very deep, very quickly. I will start with the top
layer of abstraction, then go deeper at each level, increasing in
complexity, density, and depth.

\subsection{Overview of Program}

For ease of understanding, I will assume it follows the Windows file
system form of structure.
See \ref{subsubsec:WinExec} for the file tree that is created with the program.

The goal was to utilize proper go structuring for the module. I won't
go into detail about how go is structured or about go modules. I
suggest reading \href{https://go.dev/doc/}{the go documentation} for
that information. In particular,
\href{https://go.dev/doc/modules/layout}{read here} for the exact location
on how to structure the project. Instead, I will explain how I
modeled the repo based
on this information.

Each package (eg. \textit{ui}) is contained
in its own directory, inside of the parent \textit{internal} directory.
The main function is located within the \textit{cmd} directory as
\textit{main.go}.
The \textit{tests} directory stored tests based on each package that
needed testing from the \textit{internal} directory.
The \textit{docs} directory contains the \LaTeX documentation files
as well as the exported pdf files, separated into the children
directories: \textit{LaTeX} and \textit{PDF}, respectively.

By taking a look at the structure of the repository,
\href{https://github.com/EZRA-DVLPR/GameList}{seen here},
you can see the following structure. You may notice that I omit
individual, rather unimportant, files from the structure, such as the
images within the \textit{Images} directory. This is because they are
not helpful in the overarching understanding of how the project is
laid out. They are simply a location to draw the images from.
Also, the \textit{Doc-*.tex} and \textit{Man-*.tex} are placeholder
names for the supporting files
to the .tex files that are compiled
(\textit{Documentation.tex} and \textit{Manual.tex}). Their
individual contents are once again not important to the overall
structure of the repo/project, so they are omitted.
I use the regex form of identifying files, with the '*'
meaning to collect any matching pattern (literally anything).
\href{https://regexr.com/}{See here} for an introduction to regex.

\newpage

% TODO: Match structure of this to the final version for release

\dirtree{%
	.1 ./.
	.2 cmd/.
	.3 main.go.
	.2 docs/.
	.3 Latex/.
	.4 Images/.
	.5 \dots.
	.4 Documentation.tex.
	.4 Doc-*.tex.
	.4 Manual.tex.
	.4 Man-*.tex.
	.3 PDF/.
	.4 Documentation.pdf.
	.4 Manual.pdf.
	.2 internal/.
	.3 dbhandler/.
	.4 dbhandler.go.
	.4 export.go.
	.4 import.go.
	.3 integration/.
	.4 epic.go.
	.4 gog.go.
	.4 psn.go.
	.4 steam.go.
	.3 scraper/.
	.4 scraper.go.
	.4 search.go.
	.3 ui/.
	.4 assets/.
	.5 heart.svg.
	.4 databinding.go.
	.4 dbrender.go.
	.4 newmenu.go.
	.4 popup.go.
	.4 searchbar.go.
	.4 themes.go.
	.4 toolbar.go.
	.4 ui.go.
	.2 tests/.
	.3 dbhandler\_test.go.
	.3 integration\_test.go.
	.3 scraper\_test.go.
	.2 themes/.
	.3 Dark.yaml.
	.3 Light.yaml.
	.2 .gitignore.
	.2 README.md.
	.2 go.mod.
	.2 go.sum.
}

\subsection{Structure}

The app is focused primarily into 2 places:
\begin{enumerate}
	\item The Database
	\item The UI
\end{enumerate}
just like how normal websites are separated into Backend and Frontend
(generally speaking).
The Main program calls the UI which connects the frontend to the backend.
The UI calls upon each of the other portions of the project based on user input.

\subsection{CMD - Main}

All the \textit{main.go} file does, is call the function to start the
UI. The logging
setup relies on the version number and that is contained in the
\textit{ui.go} file
which is why it isn't in \textit{main.go}.

\subsection{Docs}

To finalize the project, I wanted to complete it with Documentation
and User Manual.
The Documentation and Manual are inside the same directory since they
are both made using \LaTeX files. They can be separated further, but
they aren't meant to be read directly. Their exported versions as
PDFs are what really matter to
the end user, which is why the PDFs are in their own folder.

\subsection{Database Handling}
\label{subsec:DBHandling}

The Meat and Potatoes for the machine that runs the whole operation.
There are 3 files at work here:
\begin{enumerate}
	\item dbhandler.go
	\item export.go
	\item import.go
\end{enumerate}

\textit{dbhandler} is the main file that handles changes to the database.
Inside, it has several functions that are the core of how the
database is managed:
\begin{enumerate}
	\item CreateDB()
	\item CheckDBExists() bool
	\item DeleteAllDBData()
	\item DeleteFromDB(string)
	\item AddToDB(Game)
	\item SearchAddToDB(string, string)
	\item UpdateEntireDB()
	\item UpdateGame(string)
	\item ToggleFavorite(string)
	\item SortDB(string, bool, string)
	\item convertRowToInterface([]string) []interface{}
	\item join([]string, string) string
	\item rowsAffected(res sql.Result, string) bool
	\item compareGetGameData(Game, Game) Game
\end{enumerate}

\textit{export.go} has a runner function that calls the specified
exporting function, with the file path.
There are 3 export options:
\begin{enumerate}
	\item CSV
	\item SQL
	\item Markdown
\end{enumerate}

All exports copy the contents of the entire database and insert it
into the file at the specified location.

\textit{import.go} has a similar idea where it is a runner function
that calls the specified import with the file path.
There are 3 import options:
\begin{enumerate}
	\item CSV
	\item SQL
	\item TXT
\end{enumerate}

The SQL import drops the contents of the table before importing, making
it useful to share between devices/programs, but not to add new data
to the current database. It shrinks the size of the database by a
large factor. In my testing it downsized a 12 Kb file to
a single Kb.

The other import methods are not destructive and instead append only
new data (games/rows) to the database.

\subsubsection{Create and Check}

CreateDB() creates the DB with the table structure as follows into
the \textit{games} table:
\begin{itemize}
	\item name TEXT PRIMARY KEY,
	\item hltburl TEXT,
	\item completionatorurl TEXT,
	\item favorite INTEGER,
	\item main REAL,
	\item mainPlus REAL,
	\item comp REAL
\end{itemize}

\textit{name} is the Primary key since every game has a unique name. In the
event two games happen to have the same name, the user can alter one
of them to differentiate them.

As of version 1.0.0 there are only 2 sources of game information:
HLTB and \href{https://completionator.com/}{Completionator}.
The strings to the specific page for the game and its data are stored
for direct access when updating the game data.

\textit{favorite} is used to indicate that certain games should
always be sorted towards the top of the list. It is stored as an
integer since \href{https://www.sqlite.org/datatype3.html}{booleans
are stored directly as integers anyway} (0,1) for false and true respectively.

\textit{main}, \textit{mainPlus}, and \textit{comp} are the
categories of information that are stored in the database. They
represent the 3 ideas used by HLTB and Completionator to categorize
the time data:
How long it takes to beat the game with just the main story and
perhaps a little bit of side content, How long it takes to beat the
main story with a sizable amount of side content, and How long
it takes to beat the game with all side content completed.

CheckDB() checks if the DB exists by checking if there is a table
called 'games'.

\subsubsection{Delete from and Delete all}

DeleteAllDBData() deletes all the data from the table, but maintains
its structure.

DeleteFromDB(string) takes a game name as a parameter of type string
and executes the SQL command to delete the row from the table if it exists.

\subsubsection{Adding and Updating}

AddToDB(Game) takes a type of Game which is a struct made in the
scraper package. See \ref{subsec:Scraping} for more information
about the struct. If the received struct is verified as non-empty,
then first it checks the DB if it already exists. Recall that the
primary key is the name, so by checking the existence of the name we
can tell if there already exists an entry.
In the event it is not a duplicate, then the received Game is
inserted as a new row into the table.

SearchAddToDB(string, string) takes two arguments: The game name to
be searched, and the search source to use. The search source a value
used to determine which websites to connect to and obtain information from.
After determining which sources to select from, the game is then
queried using those sources. If there are multiple sources it
attempts to retrieve data from both source websites, and then
compares the values from both and takes the HIGHER of each value.
After obtaining the data from the source, it is temporarily stored in
a Game struct. Afterward, it calls AddToDB(Game) with the created Game.

UpdateGame(string) takes a single parameter of the game name to be updated.
The program then makes connections to the saved URLs for each source.
In the event a source doesn't have one, then the program attempts to obtain it.
The data is then compared amongst all the sources.
Afterward, the game data is completely overwritten with the data.

UpdateEntireDB() goes row by row extracting the list of names of each game.
Then it calls UpdateGame(string) with each game from the list of game names.

\subsubsection{Toggle Favorite and Sort}
\label{subsubsec:TogFavSort}

ToggleFavorite(string) flips the boolean value of favorite for the given game.

SortDB(string, bool, string) takes 3 parameters: the sorting
category, the sort order, and the query from the user.
The sorting category determines what data to sort by from the following columns:
\begin{itemize}
	\item name
	\item main
	\item mainPlus
	\item comp
\end{itemize}
The sort order is either Ascending or Descending, and determines the
order in which to sort the data by.
If there is a given query from the user, then it searches the
database for any partial matches to the string.
Otherwise, it returns the entire database sorted in the order given.
It always sorts the data with the favorites towards the top.
The output is a [][]string with all the data.

\subsubsection{Helper functions}

convertRowToInterface([]string) takes in a row as a []string and
returns the same data as a []interface{} for simpler processing.

join([]string, string) takes in a series of elements and a separator as strings.
It then conjoins the elements via the separator into a single string
that is returned.

rowsAffected(sql.Result, string) takes a result from a process on the
database as well as the name of the game the process was about.
It then checks if the result had any affected rows. If it did, then
it gets added to the log.
In both cases, it returns the boolean value of table being affected.

compareGetGameData(Game, Game) takes in two games and compares the
values for each data member of the Game struct. The larger of the two
is then used in the resulting Game to be returned.
Each link to each source is then added to the Game, but without a name.
Any function that calls the function must add the name to the
returned game from this function.

\subsection{Integrations}
\label{subsec:Integrations}

All the integration methods are similar. They make a connection to
the source, given the required information, then scrape the page for the data.

\subsubsection{Epic Games}
\label{subsubsec:EpicInt}

The epic integration requires the input JSON page (from the
instructions that need to be followed from the user), as well as the
search source.
The function parses the JSON response and obtains the list of game
names, then subsequently adds the game data to the database, given
the specified search source.

In order to obtain the data from epic, there is a series of steps
that are required:
\begin{enumerate}
	\item The user logs into Epic Games
	\item Click on their account
	\item Click "Apps \& Accounts"
	\item Open Dev Tools
	\item Go to "Network" tab in dev tools
	\item Click on the "apps" tab in the webpage
	\item Look in the dev tools for the connection to "authorized-apps"
	\item Open the page
	\item Copy the contents of the page (JSON response)
	\item Insert into the program.
\end{enumerate}

Basically, to automate this, the browser needs to look in the network
tab for the connection to the "authorized-apps" section, and parse
the JSON response.
The problem that I encountered was that I couldn't figure out a way
to do it programmatically with chromedp, without user input. Even
with all cookies input manually, the browser refused the connection
and even if it connected, there was no way to grab the JSON response.
At this point, it made more sense to leave it to be done manually.
However, if it
required user input, then there was no point in doing it in my
program. Again, the point of the program was to make the job easier, not harder.
As a result, I left it to the user to perform that task if they want
to. They can also just write each game name as a line in a TXT file
and then import it that way.
In fact, this approach only obtains games that specifically requested
permission from the user to obtain their information in the first place.
Not all games do that, and so it isn't a complete list of the games
that a user owns from epic, but I tell you that it was the best I
found and could make at that point in time.

\subsubsection{GOG}
\label{subsubsec:GOGInt}

The GOG integration requires the cookie and search source. Given the
cookie, the function attempts to make a connection to the URL with
the given cookie, and obtains the list of games. It parses the page (JSON)
and then adds each game to the database given the search source.

In essence, this is what I had hoped the Epic Games Integration would work like.

\subsubsection{PSN}
\label{subsubsec:PSNInt}

Instead of using a cookie, it needs the PSN name of the desired
user in addition to the search source. This does mean that anyone can
grab the list of games from any
other user that they can find, but this information is already public
anyways if the code succeeds in obtaining the information. This is
because the profile of the user needs to be made public in order for
PSNProfiles to obtain the data on their trophy information. I use
PSNProfiles as a source for obtaining the list of games, but once
again, it is incomplete.
It requires the user to have:
\begin{itemize}
	\item The profile be public
	\item At least some game time and perhaps 1+ trophies in the game
\end{itemize}
in order to work. I would still consider this a success though,
because the program still automates a large amount of the work.

Once the list of games is obtained from the list of pages for the
given user, each game is added into the database with the given search source.

\subsubsection{Steam}
\label{subsubsec:SteamInt}

The Steam integration is the only one that requires 2 bits of
information from the user: the profile and the cookie.
In order to avoid using the Steam API, as that would cause lots of
headache, I opted to scrape the data of public profiles again.
There seems to be a pattern here of some kind.

Besides the profile and cookie, the search source is required for
adding the list of games obtained to the database.
The general process is the same, the function makes a connection to
the page with the given information, extracts the results, then
parses the results and sends the data to the database.

\subsection{Scraping}
\label{subsec:Scraping}

The Scraping package only contains 2 files. \textit{scraper.go} is
the general one that finds the game data from the given link to HLTB
and Completionator. \textit{search.go} is used to make searches on
HLTB, Completionator, and Bing to find the URL to the given link.

Within the \textit{scraper.go} file, there is also the struct of the
Game object.
It has the same fields as those on the database:
\begin{itemize}
	\item Name
	\item HLTBUrl
	\item CompletionatorUrl
	\item Favorite
	\item Main
	\item MainPlus
	\item Comp
\end{itemize}

When calling the functions to obtain the data, it only requires the
game name to be searched.
How they work will be explained in the explanation of search.go in a
couple of paragraphs.
The goal of this function is to obtain the link from the search, then
scrape the given link for the information.
For HLTB, it removes extraneous fields, such as "All Styles" or
"Versus". For the fields "Co-Op" or "Single-Player" it obtains it and
renames it as "Main Story".
There is a precedence order to this, however:

\[
	\text{Main Story} > \text{Single-Player} > \text{Co-Op}
\]
The time data may be in a non-float format, eg. "35 1/2".
As a result, all extracted times are sent to a helper function to
turn them into appropriately formatted floats.
Once this is completed, the page scrapes the information and returns
the game struct. The process is exactly the same for Completionator.

Both of these functions require the direct URL to the game.
The URL is obtained from the \textit{search.go} file.

To search for the game with HLTB, the function queries HLTB directly for
the URL. If that doesn't yield a response, then it makes a Bing
search to try to find it.
If that fails, then an empty game struct is returned. To search
for the game with Completionator, it makes a query to Completionator.
If that doesn't work, then it returns an empty game struct.
The reason for this is Completionator isn't at the top of the
results for the search engine to obtain from. I also chose not to
search the page for the data, and instead grab the first link. There
is a timeout of 3 seconds for HLTB, Completionator, and Bing.

When making search queries, a random User Agent is selected from the
list of predefined user agents in order to avoid being blocked by
Cloudflare. Of course, I still recommend using a VPN to avoid getting
IP blacklisted.

\subsection{UI}
\label{subsec:UI}

\subsubsection{Assets}
\label{subsubsec:Assets}

The first thing in the UI section is my own Heart SVG. There is none
in the default list of icons, so I made it myself in
\href{https://inkscape.org/}{Inkscape}.
Honestly, it came out pretty good. It is slightly larger vertically
than the default icons because I didn't put a pixel border to not draw on.

\subsubsection{Rendered database as a Table}

In \textit{databinding.go}, I create a new data binding that holds
the [][]string which is the format the table data is received in to
display. The idea was to utilize a data binding to automatically
change the contents of the table whenever a change occurs to the view of the DB.

\textit{dbrender.go} handles how the DB is viewed by the user.
In it, I utilize a global variable, prevWidth, which holds the
previous width of the window. This value is compared to the current
width, and allows the table to be updated in
fixTableSize(*widget.Table, fyne.Window).
This function checks the size of the window every 0.25 seconds and
compares the previous value with the current value. If a change is
detected, then the program resizes the columns within the window.

The main function to create the table however, is handled by
createDBRender(binding.Int, binding.String, binding.Bool,
	binding.String, binding.String, *MyDataBinding,
map[string]ColorTheme, fyne.Window) and returns the *widget.Table
which was created and is able to update whenever changes occur.
Each of the bindings is input so that the function can create
listeners to change the values of the current view of the database,
then render that view to the user. Any time the user selects to
change something in the UI, whether it be the text size, sort
category, sort order, search text, selected theme, or window size,
then the table is re-rendered.

Whenever, any of the sort order, sort category, or search text are
changed, this means the view of the data in the database must change.
These changes are where the call to SortDB from dbhandler, seen
previously in section \ref{subsubsec:TogFavSort}, are made. The data
received, in the form of a [][]string, is then set to the custom-made
databinding.

UpdateTable(binding.String, binding.Int, *MyDataBinding,
binding.String, *widget.Table, float32, map[string]ColorTheme) is the
function that handles updating the contents of the table widget to
reflect the changes made in the DB view.
Whenever a change is made to the DB, the table must redefine the
dimensions, particularly the rows, as well as the contents of each
cell. Each cell is a container with 2 objects inside: a rectangle and
a label. By having each cell be a container, I have more control over
how to display the data, which is really neat for making it look
nice. The rectangle is used to denote the background color of the
cell, while the label is to be the text that is to be displayed
within the cell. The colors are dependent on the currently selected
ColorTheme. The text is a 1-to-1 correspondence with the [][]string
containing the view of the current DBs contents. Beyond the cell contents,
the headers must also be changed, to reflect the possible
change in the number of rows. Changes to the headers are handled in
headerSetup(binding.String, binding.String, *widget.Table, float32,
map[string]ColorTheme).

headerSetup defines the way that the headers are supposed to
be formatted. I cleverly used a container that contains a set of 3
different types of objects inside: A rectangle, a label, and a button.
The rectangle is the background of the header for the row numbers,
while the row numbers themselves are displayed via the label. The
buttons are used for the columns and are to change the sort Category
for the database.

Whenever the binding for the selected theme changes, the colors of
the cells and the headers should be updated.
updateTableColors(binding.Int, binding.String, *widget.Table,
map[string]ColorTheme) makes these adjustments to the color of the
cells and headers in the table.

\subsubsection{Color Theme Aware Dropdown Menu}

The next file is \textit{newmenu.go} which defines the new menu
widget, that changes colors whenever the theme changes.
This is because the default widget.PopUpMenu does not support these
changes dynamically, i.e. Once they are created, they are not changed
again.

\subsubsection{Searchbar}

I will be skipping \textit{popup.go} and discuss it in
section, \ref{subsubsec:ToolPop}. This is because it relates very
close to the toolbar, as some buttons on the toolbar open a new window.

The searchbar is defined in \textit{searchbar.go}. There is only a
single function, which returns the entire fyne.Container. This
container has 2 elements: the
Label with icon and the Entry widget for allowing user input.
I attach the Entry widget with the binding for the searchText, such
that whenever the entry changes, searchText changes.
This means that whenever the user writes or deletes any character to the widget,
the change is rendered immediately.
This is the biggest hit to
performance, but it gives the user the feedback to know the program
is working in real time.

\subsubsection{Themes}
\label{subsubsec:Themes}

The \textit{themes.go} file defines a new struct, called the
ColorTheme, which is used to update the colors across the entire
application. Because some widgets are not able to change dynamically
when the theme is changed, these widgets had to be redefined or
creatively maneuvered around with rectangles and container objects.

The loadAllThemes function takes in the file path (the themes
directory) which is checked when the app is started. When looking for
the themes, the function will read all available themes as .yaml
files. From each file, it will read the contents and
extract the colors and name of the theme.
The hexToColor function is necessary to
parse the user input colors as strings and turn them into
color.Colors, usable by the app.
The themes are separated by file and are loaded one by one in order
to be parsed as indicated by the struct. Should there be missing
Light/Dark Themes, the createLDThemes function handles creating these
default themes in the themes directory.

\subsubsection{Toolbar \& Popup}
\label{subsubsec:ToolPop}

The toolbar is spread across two files: \textit{toolbar.go} and
\textit{popup.go}. This is because the toolbar contains the buttons
that are visible to the user, while the functions for certain buttons
(those that require using another window or opening a prompt) are in
the helper file.
Within toolbar.go is the main function
createMainWindowToolbar(binding.String, binding.Bool, binding.String,
	binding.Int, *MyDataBinding, binding.String, binding.Float,
binding.String, map[string]ColorTheme, fyne.App, fyne.Window) which
returns the container containing everything in the toolbar.
The toolbar is really just a list of buttons that perform different operations.

Each button will now be described in order from left to right in the
viewing window.

\textbf{Sort} - Updates the value of the sort order binding. This
causes the text and symbol to change in the button as
well as change the current view of the database. It defaults to
whatever is saved from the user preferences. When the app runs for
the first time, it will be Ascending.

\textbf{Add} - Opens a dropdown menu that has several options:
\begin{enumerate}
	\item Single Game Name Search - Prompts the user for input for the
		game to be searched using the search source selected.
	\item Manual Entry - Prompts the user for
		the manual entries to the fields required to make the Game Struct
		and adds the given struct to the database if the given data is valid.
	\item Import from CSV - Opens the file selector to choose a CSV file
		for importing.
	\item Import from SQL - Opens the file selector to choose a SQL file
		for importing. Recall from section \ref{subsec:DBHandling} that choosing
		this will drop the current data in the database and replace it with
		the newly imported data.
	\item Import from TXT - Opens the file selector to choose a TXT file
		for importing.
	\item Import from GOG - Prompts the user for the
		"gog\_us" cookie needed for scraping the games from the user.
	\item Import from PSN - Prompts the user for the PSN profile name.
	\item Import from Steam - Prompts the user for the "sessionid"
		cookie as well as the username of the account whose games are to be scraped.
	\item Import from Epic Games - Prompts the user for the JSON data
		response to be parsed for game names.
\end{enumerate}

\textbf{Update} - If there is a non-negative row that is selected, then will
look at the sources for the given game and update the values by
accessing these sources. If the sources are not stored, then it
attempts to obtain them.

\textbf{Remove} - If there is a non-negative row that is selected, then will
delete the row from the database.

\textbf{Random} - Highlights a random row within the database.

\textbf{(Un)Favorite} - If there is a non-negative row that is
selected, then it will toggle the value of favorite for the given game.

\textbf{Export} - Opens a dropdown menu containing the following options:
\begin{enumerate}
	\item Export to CSV
	\item Export to SQL
	\item Export to Markdown
\end{enumerate}
then executes the export depending on which is selected.

\textbf{Help} - Opens a dropdown menu containing the following options:
\begin{enumerate}
	\item Video Tutorial Series - Opens the users preferred browser to
		the YouTube Playlist that contains videos on how to use the program.
	\item PDF Manual - Opens the users preferred browser to the GitHub
		repo and opens the PDF file containing the User Manual.
	\item Bug/Feature Tracker - Opens the users preferred browser to the
		GitHub repo in the issues tab.
	\item Blog Post - Opens the users preferred browser to my personal
		website where the blog post regarding this project exists.
	\item Support Me - Opens the users preferred browser to my personal
		website where the user can donate/tip me if they want to.
\end{enumerate}

\textbf{Settings} - Opens a dropdown containing the following options:
\begin{enumerate}
	\item Search Source Selector - Clicking on an option will change the
		value of the search source binding. It defaults to whatever is
		saved from the app preferences.
	\item Theme Selector - When the app is first opened, it will read
		the themes directory for all available themes as ".yaml" files.
		These files are then displayed here with a button to select it and
		update the current theme, as well as a row of boxes underneath to
		showcase a preview of the colors from the theme.
	\item Text/Icon Size Slider - Displays the size of the text and icons
		of the app. When adjusting the slider, a new label will appear
		below it to indicate the new size. When the user lets go of
		left-click, it will disappear, and all text + icon sizes will change.
	\item Update All Button - Opens a prompt to confirm the user wants
		to update all data in the database. Will do so if the user confirms.
	\item Delete All Button - Opens a prompt to confirm the user wants
		to delete all data within the database. Will do so if the user confirms.
\end{enumerate}

\subsubsection{UI.go}
\label{subsubsec:UI}

This file is the one that handles the important tasks of the app by
starting the main components.

The helper function setLogFile(string), chooses what to name the log file
that is used when the app runs.
The log file uses the provided version parameter and current
timestamp to name itself. Should
there be no logs directory, then it creates one.
It also sets all log statements to go to the logfile that was created.

The StartUI function is the one that initiates the process. It
defines what the version number is, and where the working directory
is (wherever the application is stored).
It then creates the application and main window, sets the bindings
with their default values, and reads the available themes from the
themes directory.
With this information, it creates the toolbar and search bar within a
container as well as the rendering of the table.
When the main window closes, it closes all other created windows, and
saves the current user preferences for future uses of the application.
