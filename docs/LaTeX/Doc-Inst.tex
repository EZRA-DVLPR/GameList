\section{Installation}

\subsection{Requirements}

The goal was to make a software that was Cross-Platform for desktop.
Having selected Go and \href{https://fyne.io/}{Fyne} as the language and UI
framework, I was hoping to have an easier time to export the application.
I didn't have any plans to port to mobile, but will consider it if
enough people want it. This is me, writing as of the v1.0.0 release
and I don't have precognition, so this may change.

With that being said, the requirements were support for:
\begin{itemize}
	\item Windows 10
	\item Windows 11
	\item macOS (currently 15.02 on my Macbook Pro)
	\item Linux (at the minimum Arch)
\end{itemize}

The program relies on the
\href{https://www.google.com/chrome/}{Google Chrome browser} to be installed.
I am not a fan of programs having lots of required programs and
libraries to be installed from the end user.
However, when making the web scrapers, doing it with the built-in
libraries was proving extremely difficult, so I opted to use a
different approach discussed in section \ref{subsec:Scraping}.

As such, the only requirement is to have Google Chrome installed.
If a user wants to use the integrations then that requires more
information, but isn't explicitly required for the program to function.
The different integrations and what they require are discussed in
more detail in section \ref{subsec:Integrations}.

\subsection{Installation}

Installation is extremely simple, just download the executable for
the desired OS.
Each OS handles the file structure slightly different.
The main idea behind building the program was to create a single
icon, set of metadata about
the program, and let the OS manage itself with the given instructions
of how to build for the platform from the fyne build tool.

% TODO: Check to see the trees aren't cut off on new pages
\subsubsection{Windows Executable}

This is the most straightforward. The Windows file system works as
intended in the code, which is that it creates a directory containing
the themes to read/write to. It also has another directory for the
log files. It also stores the database in the \textit{games.db} file.

The structure looks like this:

\dirtree{%
	.1 ./.
	.2 games.db.
	.2 logs/.
	.3 \dots.
	.2 themes/.
	.3 Light.yaml.
	.3 Dark.yaml.
	.3 \dots.
}

\subsubsection{macOS Application}

This one confused me for a long time.
Building with go vs fyne produced different results.
By building with go, the files and the application location were
connected but not always in the same file location, which is why I
chose to use fyne instead.
Fyne worked automatically and the files were still connected properly
no matter where the application was moved.
The problem was that I couldn't locate where the files were for fyne!
Looking deeper into the created program itself, I opened the package contents.
Then I saw that fyne had bundled the files into the app itself, and I
learned more about how macOS works with applications.

To access the files the app uses, right-click the app, and click
\textit{Show Package Contents}.
From there I will describe the file structure for the macOS application:

\dirtree{%
	.1 Contents/.
	.2 Info.plist.
	.2 MacOS/.
	.3 cmd.
	.3 logs/.
	.4 \dots.
	.2 Resources/.
	.3 games.db.
	.3 icon.icns.
	.3 themes/.
	.4 Light.yaml.
	.4 Dark.yaml.
	.4 \dots.
}

\textit{icon.icns} is the file that holds the different icon sizes on
macOS: 16x16, 32x32, 64x64, 128x128.

\textit{Info.plist} is the file that holds the metadata for the
application. It holds other information as well, but the main focus
is that it holds the version number and list of supported versions of macOS.

\subsubsection{Linux binaries}

% TODO: run and test on linux

\dirtree{%
	.1 Contents/.
	.2 Info.plist.
	.2 MacOS/.
	.3 cmd.
	.3 logs/.
	.4 \dots.
	.2 Resources/.
	.3 games.db.
	.3 icon.icns.
	.3 themes/.
	.4 Light.yaml.
	.4 Dark.yaml.
	.4 \dots.
}
